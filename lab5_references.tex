\documentclass[a4paper, 12pt]{report}

\usepackage{color}
\usepackage{graphicx}
\usepackage{subcaption}
\usepackage{amsmath} % used for showing matrix
%\usepackage{natbib}

\begin{document}

\title{My First Document}
\author{Jakir Hasan}
\date{\today}
\maketitle

\pagenumbering{roman}
\tableofcontents
\listoftables
\listoffigures
\newpage
\pagenumbering{arabic}

\setcounter{chapter}{1}

\addcontentsline{toc}{chapter}{Practical 1: Document Structure}
\chapter*{Practical 1: Document Structure}

\section{Introduction}
This is the introduction.

\section{Methods}

\subsection{Stage 1}
\label{sec1}
The first part of the methods.

\subsection{Stage 2}
The second part of the methods.

\section{Results}
Here are my results. Refering to section \ref{sec1} on page \pageref{sec1}.


\setcounter{chapter}{2}
\addcontentsline{toc}{chapter}{Practical 2: Typesetting Text}
\chapter*{Practical 2: Typesetting Text}

\section{Font Effects}

\textit{words in italic}\\
\textsl{words slanted}\\
\textsc{words in smallcaps}\\
\textbf{words in bold}\\
\texttt{words in teletype}\\
\textsf{sans serif words}\\
\textrm{roman words}\\
\underline{underlined words}

\section{Coloured Text}

{\color{red}fire} \\
{\color{red}Red}, {\color{green}green}, {\color{blue}blue}, {\color{magenta}magenta}, {\color{yellow}yellow}, and {\color{white}white}.

\section{Font Sizes}

{\tiny tiny words}\\
{\scriptsize scriptsize words}\\
{\footnotesize footnotesize words}\\
{\small small words}\\
{\normalsize normalsize words}\\
{\large large words}\\
{\Large Large words}\\
{\LARGE LARGE words}\\
{\huge huge words}


\section{Lists}

\begin{enumerate}
\item First thing
\item Second thing
\begin{itemize}
\item A sub-thing
\item Another sub-thing
\end{itemize}
\item Third thing
\end{enumerate}


\begin{itemize}
\item[-] First thing
\item[+] Second thing
\begin{itemize}
\item[Fish] A sub-thing
\item[Plants] Another sub-thing
\end{itemize}
\item[Q] Third thing
\end{itemize}

\vspace{12pt}

\section{Comments \& Spacing}

Believe that life is worth living% Note comic irony in the very first sentence
, and your belief will help create the fact.


\section{Special Characters}

\#   \$  \%  \^{}  \&  \_  \{  \}  \~{}  \textbackslash


\section{Checkpoint 2}
\#1A\textbackslash642 costs \$8 \& is sold at a \~{}10\% profit.

\setcounter{chapter}{3}
\addcontentsline{toc}{chapter}{Practical 3: Tables}
\chapter*{Practical 3: Tables}

\section{Tables}

\begin{table}
\begin{tabular}{|l|l|}
Apples & Green\\
Strawberries & Red\\
Oranges & Orange\\
\end{tabular}
\caption{Fruits}
\label{table1}
\end{table}

Table \ref{table1} shows fruits.

\vspace{12pt}

\begin{tabular}{rc}
Apples & Green\\
\hline
Strawberries & Red\\
\cline{1-1}
Oranges & Orange\\
\end{tabular}

\vspace{12pt}

\begin{tabular}{|r|c|}
\hline
8 & here's\\
\cline{2-2}
86 & stuff\\
\hline
\hline
2008 & now\\
\hline
\end{tabular}

\vspace{12pt}

\begin{tabular}{|c|c|c|c|}
\hline
\multicolumn{4}{|c|}{Country List}\\
\hline
Country Name & ALPHA 2 Code & ALPHA 3 Code & Numeric Code\\
\hline
Afghanistan & AF & AFG & 004\\
Albania & AL & ALB & 008\\
Algeria & DZ & DZA & 012\\
Angola & AO & AGO & 024\\
\hline
\end{tabular}

\newpage
\section{Checkpoint 3}

\begin{tabular}{l|r|r}
Item & Quantity & Price(\$)\\
\hline
Nails & 500 & 0.34\\
Wooden boards & 100 & 4.00\\
Bricks & 240 & 11.50\\
\end{tabular}

\vspace{12pt}

\begin{tabular}{l | c c c}
\multicolumn{1}{l |}{} &
\multicolumn{3}{c}{Year}\\
\cline{2-4}
City & 2006 & 2007 & 2008\\
\hline
London & 45789 & 46551 & 51298\\
Berlin & 34549 & 32543 & 29870\\
Paris & 49835 & 51009 & 51970\\
\end{tabular}


\setcounter{chapter}{4}
\addcontentsline{toc}{chapter}{Practical 4: Figures and Equations}
\chapter*{Practical 4: Figures and Equations}

\section{Figures}

\begin{figure}[h!]
\centering
\includegraphics[width=0.5\textwidth]{jakir_hasan}
\caption{My test image}
\end{figure}

\section{Sub Figures}

\begin{figure}[h]

\begin{subfigure}{0.4\textwidth}
\includegraphics[width=\textwidth]{Jakir}
\caption{Caption1}
\label{fig:subimage1}
\end{subfigure}

\begin{subfigure}{0.4\textwidth}
\includegraphics[width=\textwidth]{jakir_hasan}
\caption{Caption2}
\label{fig:subimage2}
\end{subfigure}

\end{figure}

\newpage

\section{Equations}

In line eqation $ 1 + 2 = 3 $. Isn't it nice?

$$ 1 + 2 = 3 $$

\begin{equation}
1 + 2 = 3
\end{equation}

\begin{eqnarray*}
	a  & = & b + c\\
  	    & = & y - z
\end{eqnarray*}

\section{Powers \& Indices}

$$ n^2 $$
$$ 2_a $$
$$ b_{a-2} $$

\section{Fractions}

$$ \frac{a}{3} $$
$$ \frac{y}{\frac{3}{x}+b}  $$


\section{Roots}

$$ \sqrt{y^2} $$
$$ \sqrt[x]{y^2} $$

\section{Sums, Limits \& Integrals}

$$ \sum_{x=1}^5y^z $$
$$ \lim_{x \to \infty} f(x) $$
$$ \int_a^b f(x)  $$


\subsection{Matrices}

\begin{equation*} % * used for unnumbered equation
	\left[
	\begin{matrix}
		1 & 0\\
		0 & 1\\
	\end{matrix}
	\right]
\end{equation*}


\subsection{Greek Letters}

$ \alpha $
$ \beta $
$\delta, \Delta$
$ \theta, \Theta $
$ \mu $
$ \pi, \Pi $
$ \sigma, \Sigma $
$ \phi, \Phi $
$ \psi, \Psi $
$ \omega, \Omega $

\subsection{Checkpoint}

\begin{eqnarray}
e = mc^2\\
\pi = \frac{c}{d}\\
\frac{d}{dx}e^x = e^x\\
\frac{d}{dx}\int_0^\infty f(s)ds = f(x)\\
f(x) = \sum_i = 0^\infty \frac{f^{(i)}(0)}{i!}x^i\\
x = \sqrt{\frac{x_i}{z}y}\\
\left[
	\begin{matrix}
		1 & 2 & 3 & 4 & 5\\
		6 & 7 & 8 & 9 & 10\\
		11 & 12 & 13 & 14 & 15\\
		16 & 17 & 18 & 19 & 20\\
		21 & 22 & 23 & 24 & 25\\
	\end{matrix}
\right]
\end{eqnarray}


\setcounter{chapter}{5}
\addcontentsline{toc}{chapter}{Practical 5: References}
\chapter*{Practical 5: References}

%I'm citing first paper\cite{Birdetal2001}\\

I'm citing second paper \cite{liang2016cnn}\\

I'm citing third paper \cite{dong2017evaluations}\\

I'm citing fourth paper \cite{hung2017applying}\\

I'm citing fifth paper \cite{bibin2017malaria}\\

I'm citing sixth paper \cite{anantharaman2018utilizing}\\

I'm citing seventh paper \cite{fuhad2020deep}\\

I'm citing nineth paper \cite{chato2017machine}\\ % natbib

I'm citing tenth paper \cite[p. 215]{liang2014deep}\\ % include page number

% multiple citation in single line
\cite{Birdetal2001, liang2014deep, fuhad2020deep}

I'm cititng chicken \cite{wikichicken}

\nocite{Birdetal2001} % Paper will appear in bibliography without in text citation.
\nocite{chang2017method}

\bibliographystyle{plain}
% \bibliographystyle{Abbrv}
% \bibliographystyle{Unsrt}
% \bibliographystyle{Alpha}
\bibliography{references}

\end{document}


